\chapter{结论与展望}
\label{cha:conclusion}

\section{结论}




知识图谱是将人类已有知识高度结构化形成的知识系统,凝结了人类千百年 积累的知识与智慧。知识图谱常被用于信息检索、问答系统和智能对话等知识驱 动的人工智能应用,辅助知识抽取、存储与推理,具有重要的实用价值与研究意 义。随着互联网时代信息的爆炸性增长,如何对知识图谱中存储的知识进行更好 地编码与表示,构建知识图谱到知识驱动的应用之间的桥梁,成为当下热门的研 究课题。为了解决计算效率与数据稀疏等问题,知识表示学习应运而生,它基于分 布式表示的思想,将实体和关系的语义信息在低维向量空间中进行表示,显著提 高了知识表示的灵活性和性能。现在,知识的分布式表示已被广泛运用于关系抽 取[58]、语言模型[59] 和问答系统[60] 等知识驱动的任务中。



\section{存在的问题与改进思路}

\section{未来的工作}







我们身处于复杂多元的世界,每时每刻都需要与各种多源信息,如文本、图 像和结构化信息等多模态信息进行交互。这些多源信息既是我们不可或缺的知识 来源,亦是我们反馈自身信息的对象与媒介。然而,传统的知识表示学习模型往 往重点关注知识图谱自身的高度结构化信息,忽略了丰富的跨模态多源信息。在 本文中,我们主要关注融合多源信息的知识表示学习,其中重点关注以下三个任 务:(1)融合实体描述信息的知识表示学习;(2)融合实体层次类型信息的知识 表示学习;(3)融合实体图像信息的知识表示学习。
在第二章,我们重点关注融合实体描述信息的知识表示学习。实体描述是对 实体自身凝练的文本描述,其中蕴含着实体各方面的丰富细节信息。这些文本信 息能够作为知识图谱结构化信息的补充,帮助构建更准确的知识表示。然而,联合 考虑实体描述面临着如何自动抽取描述中高质量的文本信息,以及如何融合结构 信息与文本信息进行联合学习等挑战,已有引入文本信息的知识表示模型也仅仅 孤立地考虑词级别的文本信息,而忽略了篇章级别语序语义信息的影响。为了解 决这些问题,我们基于平移假设提出了融合实体描述的知识表示学习模型,为每 个实体设置了基于结构与基于描述的两种表示,并使用神经网络模型对实体描述 进行建模。实验结果表明,我们的模型能够充分利用实体描述中的文本信息,提升 知识图谱补全和实体类型分类等任务的效果,在对新实体的知识表示上也有较好 的表现。
在第三章,我们重点关注融合实体层次类型信息的知识表示学习。实体层次 类型指的是实体所属不同粒度的类型信息,这些类型往往储存在层次化的结构中。
63
第 5 章 总结与展望
 实体类型信息能够帮助人类构建层次化的认知体系,提供实体结构化的先验知识, 也能暗示实体在不同情境下更应表现出的类型。然而,传统知识表示模型较少考 虑实体类型信息,也未能充分利用实体类型的层次结构信息。为了解决这些问题, 我们设计了融合实体层次类型的知识表示学习模型,提出实体在不同关系下应该 突出不同实体类型,并具有类型特化的实体表示。我们使用两种层次类型编码器 对实体类型的层次结构进行建模,构建映射矩阵,同时在训练与测试中进行了实 体类型限制,进一步提高知识表示的性能。实验结果表明,我们的模型能够充分利 用实体层次结构的类型信息,提升知识图谱补全和三元组分类等任务的效果,并 在长尾分布的数据集上也能得到超过基线模型的表现。
在第四章,我们重点关注融合实体图像信息的知识表示学习。实体图像能够 提供实体外形、行为和其它相关实体的视觉信息,可以帮助模型全方位地理解实 体。融合图像信息与知识图谱结构信息构建跨模态的知识表示,既能通过图像信 息提高知识表示的性能,也能将知识引入图像领域,为图像与知识的联合应用提 供基础。由于图像信息与知识信息在储存与表示上存在较大差异,如何联合两个 异质空间进行学习成为此任务的最大的挑战。另外,海量的图像信息质量良莠不 齐,选取的图像质量对知识表示结果也有较大影响。针对这些问题,我们提出了融 合实体图像的知识表示学习模型,构建实体基于描述和基于图像的两种知识表示。 具体地,我们使用基于神经网络模型的图像表示模块和图像映射模块构建图像特 征在知识空间的表示,然后引入注意力机制自动选择高质量的实体图像构建实体 基于图像的表示。实验结果证实了实体图像中的视觉信息能够帮助模型构建更准 确的知识表示,在知识图谱补全和三元组分类等任务上都表现出更好的效果。经 过实例分析,我们也证实了模型在图像-知识联合空间上的语义平移现象,以及注 意力机制对模型效果的正面影响。